% Options for packages loaded elsewhere
\PassOptionsToPackage{unicode}{hyperref}
\PassOptionsToPackage{hyphens}{url}
\PassOptionsToPackage{dvipsnames,svgnames*,x11names*}{xcolor}
%
\documentclass[
  ignorenonframetext,
]{beamer}
\usepackage{pgfpages}
\setbeamertemplate{caption}[numbered]
\setbeamertemplate{caption label separator}{: }
\setbeamercolor{caption name}{fg=normal text.fg}
\beamertemplatenavigationsymbolsempty
% Prevent slide breaks in the middle of a paragraph
\widowpenalties 1 10000
\raggedbottom
\setbeamertemplate{part page}{
  \centering
  \begin{beamercolorbox}[sep=16pt,center]{part title}
    \usebeamerfont{part title}\insertpart\par
  \end{beamercolorbox}
}
\setbeamertemplate{section page}{
  \centering
  \begin{beamercolorbox}[sep=12pt,center]{part title}
    \usebeamerfont{section title}\insertsection\par
  \end{beamercolorbox}
}
\setbeamertemplate{subsection page}{
  \centering
  \begin{beamercolorbox}[sep=8pt,center]{part title}
    \usebeamerfont{subsection title}\insertsubsection\par
  \end{beamercolorbox}
}
\AtBeginPart{
  \frame{\partpage}
}
\AtBeginSection{
  \ifbibliography
  \else
    \frame{\sectionpage}
  \fi
}
\AtBeginSubsection{
  \frame{\subsectionpage}
}
\usepackage{lmodern}
\usepackage{amssymb,amsmath}
\usepackage{ifxetex,ifluatex}
\ifnum 0\ifxetex 1\fi\ifluatex 1\fi=0 % if pdftex
  \usepackage[T1]{fontenc}
  \usepackage[utf8]{inputenc}
  \usepackage{textcomp} % provide euro and other symbols
\else % if luatex or xetex
  \usepackage{unicode-math}
  \defaultfontfeatures{Scale=MatchLowercase}
  \defaultfontfeatures[\rmfamily]{Ligatures=TeX,Scale=1}
\fi
\usetheme[]{Rochester}
\usecolortheme{seagull}
\usefonttheme{professionalfonts}
% Use upquote if available, for straight quotes in verbatim environments
\IfFileExists{upquote.sty}{\usepackage{upquote}}{}
\IfFileExists{microtype.sty}{% use microtype if available
  \usepackage[]{microtype}
  \UseMicrotypeSet[protrusion]{basicmath} % disable protrusion for tt fonts
}{}
\makeatletter
\@ifundefined{KOMAClassName}{% if non-KOMA class
  \IfFileExists{parskip.sty}{%
    \usepackage{parskip}
  }{% else
    \setlength{\parindent}{0pt}
    \setlength{\parskip}{6pt plus 2pt minus 1pt}}
}{% if KOMA class
  \KOMAoptions{parskip=half}}
\makeatother
\usepackage{xcolor}
\IfFileExists{xurl.sty}{\usepackage{xurl}}{} % add URL line breaks if available
\IfFileExists{bookmark.sty}{\usepackage{bookmark}}{\usepackage{hyperref}}
\hypersetup{
  pdftitle={Introduction to Mathematical Optimization and Mixed-Integer Programming},
  pdfauthor={Yuanzhe(Roger) Li},
  colorlinks=true,
  linkcolor=Maroon,
  filecolor=Maroon,
  citecolor=Blue,
  urlcolor=blue,
  pdfcreator={LaTeX via pandoc}}
\urlstyle{same} % disable monospaced font for URLs
\newif\ifbibliography
\setlength{\emergencystretch}{3em} % prevent overfull lines
\providecommand{\tightlist}{%
  \setlength{\itemsep}{0pt}\setlength{\parskip}{0pt}}
\setcounter{secnumdepth}{-\maxdimen} % remove section numbering

\title{Introduction to Mathematical Optimization and Mixed-Integer Programming}
\author{Yuanzhe(Roger) Li}
\date{11/12/2019}

\begin{document}
\frame{\titlepage}

\begin{frame}{Overview}
\protect\hypertarget{overview}{}

\begin{itemize}
\tightlist
\item
  Mathematical optimization
\item
  Mixed-integer programming

  \begin{itemize}
  \tightlist
  \item
    Traveling salesman problem
  \item
    Portfolio optimization
  \end{itemize}
\end{itemize}

\end{frame}

\begin{frame}{Overview of mathematical optimization}
\protect\hypertarget{overview-of-mathematical-optimization}{}

\end{frame}

\begin{frame}{Mixed-Integer Programming (MIP)}
\protect\hypertarget{mixed-integer-programming-mip}{}

\[\begin{aligned}
\max \quad & f(x)\\
s.t. \quad & g(x) \leq 0 \\
& x \in \mathbb{R} \times \mathbb{Z}
\end{aligned}\]

\end{frame}

\begin{frame}{Portfolio Optimization}
\protect\hypertarget{portfolio-optimization}{}

\end{frame}

\begin{frame}{Traveling salesman problem (TSP)}
\protect\hypertarget{traveling-salesman-problem-tsp}{}

\begin{itemize}
\tightlist
\item
  \href{https://en.wikipedia.org/wiki/Travelling_salesman_problem}{TSP
  wikipedia page}
\end{itemize}

\begin{quote}
The travelling salesman problem (TSP) asks the following question: Given
a list of cities and the distances between each pair of cities, what is
the shortest possible route that visits each city exactly once and
returns to the origin city?
\end{quote}

\begin{center}\includegraphics{math_opt_slides_files/figure-beamer/tsp_setup-1} \end{center}

\end{frame}

\begin{frame}{Formulate TSP as a MIP}
\protect\hypertarget{formulate-tsp-as-a-mip}{}

\end{frame}

\end{document}
